\documentclass[a4paper,11pt]{article}

\usepackage[czech]{babel}
\usepackage[left= 1.5cm,text={18cm, 25cm},top=2.5cm]{geometry}
\usepackage[utf8]{inputenc}
\usepackage{times}
\usepackage{paralist}
\usepackage{graphicx}
\usepackage{textcomp}
\usepackage{enumitem}
\usepackage{amssymb}
\usepackage{amsmath}
\usepackage{xcolor}
\usepackage[ddmmyyyy]{datetime}
\usepackage{array}
\pagestyle{plain}
\pagenumbering{arabic}

\renewcommand*\contentsname{Obsah}

\newdateformat{mydate}{\twodigit{\THEDAY}.{ }\shortmonthname[\THEMONTH] \THEYEAR}

\pagenumbering{arabic}

\usepackage{url}
\DeclareUrlCommand\url{\def\UrlLeft{<}\def\UrlRight{>} \urlstyle{tt}}

% \usepackage{indentfirst}

\begin{document}
\selectlanguage{czech}

\begin{titlepage}
\begin{center}
    {\Huge \textsc{Vysoké učení technické v Brně}}
\vspace{\stretch{0.01}}
    
    {\LARGE \uppercase{FAKULTA INFORMAČNÍCH TECHNOLOGIÍ}}
    
\begin{figure}[h]
\vspace{5.0cm}
\centering
\includegraphics[scale=0.15]{logo.png}
\vspace{-10.0cm}
\end{figure}
    
\vspace{\stretch{0.382}}
	{\LARGE Projekt IAL, 2018Z}
\vspace{\stretch{0.02}}

	{\Huge \textbf{Obarvení grafu}}
\vspace{\stretch{0.02}}\\

{\LARGE {Projekt č.6}}\\

\begin{figure}[h]
\centering
{\Large {\mydate\today}}
\vspace{6cm}
\end{figure}

\end{center}
\begin{compactitem}
\item[] \textbf{Tým:}
\item[] Adámek Josef, xadame42
\item[] Barnová Diana, xbarno00
\item[] Vanický Jozef, xvanic09
\item[] Weigel Filip, xweige01
\end{compactitem}

\end{titlepage}

\tableofcontents
\newpage

\section{Zadanie}
Obarvením grafu rozumíme přiřazení barev uzlům grafu, přičemž žádné dva sousední uzly nesmí být obarveny stejně. Minimální počet použitých barev se nazývá chromatické číslo. 


Vytvořte program pro hledání \textbf{minimálního obarvení neorientovaných grafů.} 
Pokud existuje více řešení, stačí nalézt pouze jedno. Výsledky prezentujte vhodným způsobem. Součástí projektu bude načítání grafů ze souboru a vhodné testovací grafy. V dokumentaci uveďte teoretickou složitost úlohy a porovnejte ji s experimentálními výsledky.

\section{Průzkum kontextu použití}
\subsection{Cílová skupina}
TBD

\subsection {Typické případy použití}


\subsection{Prostředí použití}
TBD

\subsection{Požadavky na produkt}
TBD
\section{Návrh klíčových prvků UI}
TBD

\subsection{Seznam projektů}
TBD

\subsection{ Popis projektu}
TBD

\section{Návrh GUI a Prototyp}
TBD

\section{Testování prototypu GUI}
TBD

\section{Studijní zdroje}
TBD

\section{Implementace}
TBD

\section{Týmová spolupráce}
TBD

\section{Závěr}
TBD


\end{document}